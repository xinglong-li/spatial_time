\section{Introduction}
In practice, it is common that the selection of locations of sites where the pollutants 
are monitored are affected by the density of the pollutants. It is crucial to take the preferential 
sampling effect into account to accurately model the dispersion of the pollutant and to make 
predictions of pollutants either spatially or into the future.

\cite{Watson2019_pref_samp} proposed a framework that jointly modeling the distribution of an
environmental process and a site-selection process, where the environmental process can be spatial,
temporal, or spatio-temporal. By sharing the random effects between the two process, the joint
model can detect the preferential sampling effects in site selection. 

In this work, we develop an R package for this joint model framework for the purpose of making
spatial predictions. We demonstrate this R package by applying it to the modeling and prediction of
PM10 distributions in the south coast air basin(SOCAB) region in California. 