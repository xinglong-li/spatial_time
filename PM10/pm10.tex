\section{The Preferential Sampling Model}
The population of sites considered for selection should also be selected carefully.
Different choices of the population leads to different conclusions about the PS effect. 
In one case the population is all sites that have been monitored at some times $t \in T$, and
the estimate of the mean value of the PM10 can be interpreted as the network average.
By using this population, the model help us detect the effect of PS on estimates of the density of 
PM10s across all sites ever observed. 
In the other case, we include all vertices of the mesh grid that are inside the border in the 
population and we treat those unobserved vertices as pseudo site locations. These pseudo sites are
placed at a density of approximately 3 km throughout SOCAB region, and in this case, the estimate 
of the mean value of the PM10 in this case can be interpreted as the PM10 density across the SOCAB
region. Since we are uniformly cover the 
SOCAB region, this population help us detect if the observed sites are preferentially selected and 
the effect of PS on estimating the mean of PM10 over the entire SOCAB region.

A Bayesian model is 
introduced for the joint distribution of the response vector $(Y_{st}, R_{st})$, where $R_{st}$ is
a binary response for the site selection process. By sharing random effects across the two processes
the stochastic dependence between the observation and the site selection can be detected and adjust 
the predictions. In particular, the site selection process is allowed to use information from both
spatially varying Gaussian processes and spatially-uncorrelated site-specific effects to determine
the site selection probabilities each year.
We fit the same preferential sampling model for the two populations.


\section{PM10 in California}
The annual concentration of PM10s from 1965 can be download from **. We download the site locations
and the observed PM10 concentrations from the website. The site locations of these sites can be 
seen from Fig. 

The decline trend in concentrations of PM10s from 1965 to can be seen from Fig. The sites are added 
to the network and dropped. It can be seen from the plot that sites remained in the network until 
the end are those with higher measurements.
\subsection{The PM10 Data}
A few data cleaning steps were carried out before fitting the models. Due to the right skewness of 
the PM10 observation distribution, we applied the natural logarithmic transformation to the values
to make the observation more Gaussian in shape. Before the log transformation, we firstly divide
each value by mean of all recorded values to make the response dimensionless.
We scale the Eastings and Northings coordinates and the unit is 100 km. We scaled the years to
lie in the interval $[0, 1]$ to stabilize the temporal polynomials used in later analysis.

\subsection{Data Preprocessing}

\subsection{Map projection and Mesh Grid}
The same mesh is used in both implementations. 

\section{Model Fitting}
We fit the same model on two populations using R-inlabru package. Inlabru is built upon the R-INLA 
package with simplified syntax. The R-INLA package apply the SPDE approach to add the
This enables the rapid computation of approximate Bayesian posterior distribution of the model 
parameters and random effects.
