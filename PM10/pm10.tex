\section{PM10 in California}

\subsection{The data}
A few data cleaning steps were carried out before fitting the models. Due to the right skewness of 
the PM10 observation distribution, we applied the natural logarithmic transformation to the values
to make the observation more Gaussian in shape. Before the log transformation, we firstly divide
each value by mean of all recorded values to make the response dimensionless.
We scale the Eastings and Northings coordinates and the unit is 100 km. We scaled the years to
lie in the interval $[0, 1]$ to stabilize the temporal polynomials used in later analysis.

\subsection{Modeling}
We build one model from the general framework introduced earlier. Let $t_j^{\star}$ denote 
the $j$th time-scaled observations that lie in the interval $[0, 1]$.

The model for the observation process is
\[
Y_{i, j} \given R_{i, j} &\sim \distNorm(\mu_{i, j}, \sigma_{\epsilon}^2) \\
\mu_{i, j} &= \gamma_0 + \gamma_1 t_j^{\star} + \gamma_2(t_j^{\star})^2 
                       + b_{0, i} + b_{1, i}t_j^{\star} 
                       + \beta_0(s_i) + \beta_1(s_i)t_j^{\star} + \beta_2(s_i)(t_j^{\star})^2 \\
[\beta_k(s_1), \ldots, \beta_k(s_m)]^T &\distiid \distNorm(0, \Sigma(\zeta_k)) \quad 
\text{for}\ k \in \{0, 1, 2\}, \quad \Sigma(\zeta_k) = \text{Matern}(\zeta_k) \\
[b_{0, i}, b_{1, i}]  &\sim \distiid \distNorm(0, \Sigma_b), \quad 
\Sigma_b = \bmat \sigma_{b, 1}^2 & \rho_b  \\ \rho_b & \sigma_{b, 2}^2 \emat, \\ 
\theta &= (\sigma_{\epsilon}^2, \gamma, \zeta_k, \sigma_{b, 1}^2, \rho_b) \sim \text{Priors}.
\]
The sources of variation can be broken into three components. 
$\gamma_0 + \gamma_1 t_j^{\star} + \gamma_2(t_j^{\star})^2$ is global variation, 
$b_{0, i} + b_{1, i}t_j^{\star}$ is independent site-specific variation and
$\beta_0(s_i) + \beta_1(s_i)t_j^{\star} + \beta_2(s_i)(t_j^{\star})^2$ is smooth spatially correlated
variation. To ensure model identifiability, we enforce sum-to-zero constraints on all random effects
($\beta$ and $b$), and we do not estimate spatially-uncorrelated random effects $b$ at locations
with no observations. In the notation of the general framework, $q_1 = 5$. The independent 
site-specific variations are captured by the IID random intercepts and random slopes 
$(b_{0, i}, b_{1, i})$. 

The model for site-selection process is 
\[
R_{i, j} &\sim \distBern(p_{i, j}) \\
\mathrm{logit} p_{i, 1} &= \alpha_{0, 0} + \alpha_1 t_1^{\star} + \alpha_2(t_1^{\star}) 
 + \beta_1^{\star}(t_1)  \\
& + \alpha_{rep} I_{i, 2} + \beta_0^{\star}(s_i)  \\
& + d_b[b_{0, i} + b_{1, i}(t_1^{\star})] \\
& + d_{\beta}[\beta_0(s_i) + \beta_1(s_i)t_{j-1}^{\star} + \beta_2(s_i)(t_{j-1}^{\star})^2], \\
\mathrm{for} j \ne 1 \quad \mathrm{logit} p_{i, j} &= \alpha_{0, 1} + \alpha_1 t_j^{\star} + 
\alpha_2 (t_j^{\star})^2 + \beta_1^{\star} t_j \\
& + \alpha_{ret}r_{i, (j-1)} + \alpha_{rep} I_{i, 2} + \beta_0^{\star}(s_i)  \\
& + d_b[b_{0, i} + b_{1, i}(t_1^{\star})] \\
& + d_{\beta}[\beta_0(s_i) + \beta_1(s_i)t_{j-1}^{\star} + \beta_2(s_i)(t_{j-1}^{\star})^2], \\
& I_{i, j} = \ind\left[ \left( \sum_{\ell\ne i}r_{\ell, j-1}\ind (\|s_i - s_{\ell}\| < c) \right) > 0 \right],\\
[\beta_0^{\star}(s_1), \ldots, \beta_0^{\star}(s_m)]^T &\sim \distNorm(0, \Sigma(\zeta_R)),
\Sigma(\zeta_R) = \mathrm{Matern}(\zeta_R), \\
[\beta_1^{\star}(t_1), \ldots, \beta_1^{\star}(t_T)]^T &\sim \mathrm{AR1}(\rho_a, \sigma_a^2), \\
\theta_R = [\alpha, d_b, d_{\beta}, \rho_a, \sigma_a^2, \zeta_R] & \sim \mathrm{Priors}
\]

The first component is the global effects of time on the log odds of selection. We also add 
first-order autoregressive deviation, $\beta_1^{\star}(t_j)$, from this global quadratic change.
$\alpha_{ret}$ represents the "retention effect" reflecting how the probability a site is selected in a 
given year changes, conditioned on its inclusion in the previous year. Here, we share all parameters
across the two processes and allow only a unique intercept to exist between the processes.
$\alpha_{rep}$ captures the repulsion effect. $I_{i, j}$ denote an indicator variable that determines
whether or not another site in the network placed within a distance $c$ from site $i$ was
was operational at the previous time $t_{j-1}$. We choose the hyperparameter $c$ to be 10 km.

This is a joint model with three processes: an observation process, an initial site-placement process
and a site-retention process. We only allow for a unique intercept to exist across the two processes, 
sharing the remaining parameters. Only the pseudo-sites contribute a zero to the 
Bernoulli likelihood for the site-placement across all years. Only the sites that have been removed 
from the network in year $j$ contribute a zero to the Bernoulli likelihood for the site-retention
process at year $j$. This ensures that no site in the network was ever reinstalled after its removal. 
