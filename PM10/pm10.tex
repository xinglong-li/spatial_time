\section{The Preferential Sampling Model}
The population of sites considered for selection should also be selected carefully.
Different choices of the population leads to different conclusions about the PS effect. 
In one case the population is all sites that have been monitored at some times $t \in T$, and
the estimate of the mean value of the PM10 can be interpreted as the network average.
By using this population, the model help us detect the effect of PS on estimates of the density of 
PM10s across all sites ever observed. 

In the other case, we include all vertices of the mesh grid that are inside the border in the 
population and we treat those unobserved vertices as pseudo site locations. These pseudo sites are
placed at a density of approximately 3 km throughout SOCAB region, and in this case, the estimate 
of the mean value of the PM10 in this case can be interpreted as the PM10 density across the SOCAB
region. Since we are uniformly cover the 
SOCAB region, this population help us detect if the observed sites are preferentially selected and 
the effect of PS on estimating the mean of PM10 over the entire SOCAB region.

\section{PM10 in California}
The annual concentration of PM10s from 1965 can be download from the website (\url{https://www.epa.gov/outdoor-air-quality-data}) 
of the U.S. Environmental Protection Agency (EPA). We download the annual records of PM10 in California 
between 1985 to 2022. The raw data set include locations, year, and some summery statistics of 
measurements of all sites in California. The complete information of the data set can be found in
the EPA website (\url{https://aqs.epa.gov/aqsweb/airdata/FileFormats.html#_annual_summary_files}).
The raw data set downloaded also include records of other air pollutants, but we keep only the PM10
records.

We keep the annual mean of PM10 measurements to represent the PM10 level at each site.
Sometimes exceptional events happened and can affect the measurements of air pollutants, 
but the local agency has no control over. A wildfire is an example of an exceptional event. 
We use the summary statistics which remove the affects of extreme events.  

The site locations of these sites can be seen from \cref{fig:socab_sites}. Note that each measurement site might has 
multiple monitors planted in close but different locations. We combine the measurement of different
monitors of each site by taking the arithmetic average of both the locations and PM10 measurements.
\begin{figure}[ht]
	\centering
	\includegraphics[width = 0.8\textwidth]{socab_plots/SOCAB_sites.png}
	\caption{The sites in the SOCAB region. Each blue dot represent a site. If a site has multiple monitors, the site location is determined by taking the average of coordinates of all monitors.}
	\label{fig:socab_sites}
\end{figure}

The decline trend in concentrations of PM10s from 1985 to 2022 can be seen from \cref{fig:logpm10_traces}. The sites are added 
to the network and dropped. It can be seen from the plot that sites remained in the network until 
the end are those with higher measurements.
The trend of variance of $\log$(PM10) can be seen from \ref{fig:logpm10_var}. 
\subsection{The PM10 Data}
A few data cleaning steps were carried out before fitting the models. Due to the right skewness of 
the PM10 observation distribution, we applied the natural logarithmic transformation to the values
to make the observation more Gaussian in shape. Before taking the log transformation, we firstly divide
each value by mean of all recorded values to make the response dimensionless.
We scale the East and North coordinates and the unit is 10 km. We scaled the years to
lie in the interval $[0, 1]$ to stabilize the temporal polynomials used in later analysis.

\subsection{Data Preprocessing}
In order to make sure the the assumptions on the distributions of data is reasonable, some data
cleaning and preprocessing is required before we fit the PS model. Due to the right skewness of the
PM10 observations, we applied the natural logarithmic transformation to the values to make the 
observations Gaussian distributed. To make the fitted model interpretable, we then subtract the 
logarithmic transformation of the mean value so that the data is dimensionless. 
\begin{figure}[ht]
	\centering
	\includegraphics[width = 0.8\textwidth]{socab_plots/logPM10_traces.png}
	\caption{The sites in the SOCAB region. Each curve represents a measurement history of one site.}
	\label{fig:logpm10_traces}
\end{figure}

\begin{figure}[ht]
	\centering
	\includegraphics[width = 0.8\textwidth]{socab_plots/logPM10_var.png}
	\caption{The sites in the SOCAB region.}
	\label{fig:logpm10_var}
\end{figure}

\subsection{Map projection and Mesh Grid}
The site locations in the data set are recorded as latitude and longitude under different coordinate
reference systems (CRS). In order to better represent the distance between sites, we project all 
site locations to the UTM (Easting/Northing) coordinates with the measurement unit being kilometer.

The border map of SOCAB region is also projected to the same CRS as the site locations, and we keep
the sites only in the SOCAB region. 

\begin{figure}[ht]
	\centering
	\includegraphics[width = 0.8\textwidth]{socab_plots/SOCAB_meshgrid.png}
	\caption{The meshgrid in the SOCAB region.}
	\label{fig:socab_meshgrid}
\end{figure}

To increase the numerical stability in model fitting, we rescale the Eastings and Northings 
coordinates of sites and the SOCAB border by 10, and each unit distance represent 10 km.

We create the mesh grid using the function $mesh_2d_inla$. 

The same mesh is used in both implementations. 


\section{Model Fitting}
In spatial statistics it is common to formulate mixed-effects regression models in which the linear predictor is made of a trend plus a spatial variation
The trend usually is composed of fixed effects or some smooth terms on covariates, while the spatial variation is usually modeled using correlated random effects.
Spatial random effects often model (residual) small scale variation and this is the reason why these models can be regarded as models with correlated errors.

Lindgren et al. (2011) describe an approximation to continuous spatial models with a Matérn
covariance that is based on the solution to a stochastic partial differential equation (SPDE). 
A Gaussian spatial process with Matérn covariance is a solution to SPDE
This approximation is computed using a sparse representation that can be effectively implemented using the integrated nested Laplace approximation (INLA, Rue et al., 2009)

INLA focuses on models that can be expressed as latent Gaussian Markov random fields (GMRF)

INLA can handle models with more than one likelihood. By using a model with more than one likelihood it is possible to build a joint model with different types of outputs and the hyperparameters in the likelihoods will be fitted separately.

We fit the same model on two populations using R-inlabru package. Inlabru is built upon the R-INLA 
package with simplified syntax. The R-INLA package apply the SPDE approach to add the
This enables the rapid computation of approximate Bayesian posterior distribution of the model 
parameters and random effects. The R-INLA packages approximates the Gaussian Markov random field by
solving an SPDE on a triangulation grid.
The goal of inlabru is to facilitate spatial modeling using integrated nested Laplace approximation via the R-INLA package.

The data set \textbf{PM10s\_SOCAB} has $35 \times 38$ ($\text{number of site} \times \text{number of years}$) 
rows, where each row represent the measurement of one site in a given year. If a site were not
measured in some years, the measurement was noted as missing.  The following variables:
\begin{itemize}
	\item \textbf{annual\_mean}: The logarithmic transformation of annual mean value of PM10s, which acts as the response variable in the regression model.
	\item \textbf{slc}: A dummy variable (0 or 1) indicating whether a site was selected in each year.
	This is the response variable in the site selection model.
	\item \textbf{slc\_lag}: A dummy variable indicating whether a site was selected in last year.
	\item \textbf{site\_number}: The indicator of each site, which is used as the group indicator for random effects
	\item \textbf{locs}: The North/East coordinates (unit 10 km) of sites.
	\item \textbf{year}: The year in which a observation was recorded.
	\item \textbf{time}: The standardized years. 
	\item \textbf{repulsion\_ind}: A dummy variable indicating whether there was other sites in the close neighbor of a site last year.
	\item \textbf{zero}: A vector of zeros that is used as the auxiliary variable in the auxiliary models.
\end{itemize}

We fit the joint model using the R-inlabru package. Our joint model includes four likelihoods, which
includes a Gaussian likelihood for the PM10 concentration process, a binomial likelihood for the
site selection process, and two auxiliary Gaussian likelihoods that are introduced to share linear
combinations of random effects across the PM10 concentration process and the site selection process.

According to the syntax of the R-inlabru package, all components (including the shared ones) of all
likelihoods need to be firstly claimed at once and then used in defining the likelihoods. Each 
effect is defined using a user-assigned name, the variable, and the random distribution. For example


\begin{lstlisting}[language = R]
	components <- ~ 
		# Components for observation model
		intercept_obs(1) +   
		time_1_obs(time) +  
		time_2_obs(time^2) +  
		random_0_obs(site_number, model = "iid2d", n = no_sites*2, constr=TRUE) +  
		random_1_obs(site_number, weights = time, copy = "random_obs0") +  
		spatial_0_obs(locs, model = spde_obj) +  
		spatial_1_obs(locs, weights = time, model = spde_obj) +  
		spatial_2_obs(locs, weights = time^2, model = spde_obj) +    
		# Components for site selection model  
		intercept_slc(1) + 
		time_1_slc(time) +  
		time_2_slc(time^2) +  
		lag_slc(slc_lag) +  
		repuls_slc(repulsion_ind) + 
		ar_slc(year, model='ar1', hyper=list(theta1=list(prior="pcprec",param=c(2, 0.01)))) +  
		spatial_slc(locs, model = spde_obj) +  
		share_aux1(site_number, copy = "comp_aux1", fixed = FALSE) +   
		share_aux2(site_number, copy = "comp_aux2", fixed = FALSE) +    
		# Components for the first auxiliary model  
		random_0_aux1(site_number, copy = "random_obs0", fixed = TRUE) +  
		random_1_aux1(site_number, weights = time, copy = "random_obs11", fixed = TRUE) +  
		comp_aux1(site_number, model = 'iid', 
							hyper = list(prec = list(initial = -20, fixed=TRUE))) +    
		# Components for the second auxiliary model  
		spatial_0_aux2(locs, copy = "spatial_0_obs", fixed = TRUE) +  
		spatial_1_aux2(locs, weights = time, copy = "spatial_1_obs", fixed = TRUE) +  
		spatial_2_aux2(locs, weights = time^2, copy = "spatial_2_obs", fixed = TRUE) +  
		comp_aux2(site_number, model = 'iid', 
							hyper = list(prec = list(initial = -20, fixed=TRUE)))
\end{lstlisting} \label{code:inlabru_components}
\paragraph{The observation likelihood}
We assume that the concentration of PM10s in SOCAB follow a normal distribution. The mean value of 
the concentration is modeled as a combination of a fixed effect and random effects. The fixed 
effect is a quadratic function of time, and the random effects include spatially correlated process,
which is a Gaussian Markov random field, and a site-specific random effect.  

\begin{lstlisting}[language = R]
	like_obs <- like(formula = annual_mean ~ intercept_obs + time_1_obs + time_2_obs +     
																					 random_0_obs + random_1_obs + 
																					 spatial_0_obs + spatial_1_obs + spatial_2_obs,  
									 family = "gaussian",  
									 data = PM10s_SOCAB)
\end{lstlisting} \label{code:inlabru_lik_obs}

The site selection is assumed to follow a Binomial distribution, which is connected to the linear
predictor via a logistic transformation. The fixed effect include a quadratic term and terms 
indicating a site was selected a year before, and a variable indicating the presence of other sites
within certain distance from a site. The random effects including a spatially correlated effect,
and a temporally correlated effect. In order to detect the preferential sampling effect, the random
effects in the observation model are added to the observation process. The shared effects across two
models allow for the stochastic dependence between the two models.

\begin{lstlisting}[language = R]
	like_slc <- like(formula = slc ~ intercept_slc + time_1_slc + time_2_slc +     
																	 lag_slc + repuls_slc + ar_slc + spatial_slc +     
																	 share_aux1 + share_aux2,  
									 family = "binomial",  
									 Ntrials = rep(1, times = length(PM10s_SOCAB$slc)),  
									 data = PM10s_SOCAB)
\end{lstlisting} \label{code:inlabru_lik_slc}

While the R-INLA package and the R-inlabru package allows for components sharing across models, 
sharing of the linear combination of multiple components is not straightforward. To share a linear
combination of components between the two likelihoods, we introduce an auxiliary model with zero 
mean and a vary large variance, and an auxiliary variable to copy the (negative) joint effect of 
the linear combination of multiple effects. 

\begin{lstlisting}[language = R]
	like_aux1 <- like(formula = zero ~ random_0_aux1 + random_1_aux1 + comp_aux1,  
										family = "gaussian",  
										data = PM10s_expand)
\end{lstlisting} \label{code:inlabru_lik_aux1}

\begin{lstlisting}[language = R]
	like_aux2 <- like(formula = zero ~ spatial_0_aux2 + spatial_1_aux2 + spatial_2_aux2 + 
																		 comp_aux2,  
										family = "gaussian",  
										data = PM10s_SOCAB)
\end{lstlisting} \label{code:inlabru_lik_aux2}

\begin{lstlisting}[language = R]
	bru_options_set(bru_max_iter = 20,                
									control.inla = list(strategy = "gaussian", int.strategy = 'eb'),                
									control.family = list(
											list(), 
								    	list(), 
								     	list(hyper = list(prec = list(initial = 20, fixed=TRUE))),                  
								     	list(hyper = list(prec = list(initial = 20, fixed=TRUE)))),                
									bru_verbose = T)
	fit_bru <- bru(components, 
								 like_obs, like_slc, like_aux1, like_aux2)
\end{lstlisting} \label{code:inlabru_fit}

\section{Preferential Sampling Effects}