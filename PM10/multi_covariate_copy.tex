\documentclass{article}
\usepackage[margin=1in]{geometry}
\usepackage[autonum]{tchdr}
\usepackage[sort&compress]{natbib}


\title{Copy Multiple Features in INLA / InlaBru}
%\author{Xinglong Li \\xinglong.li@stat.ubc.ca}

\begin{document}
\maketitle

\section{Background}
\cite{Watson2019_pref_samp} proposed a framework that jointly modeling the distribution of an
environmental process and a site-selection process, where the environmental process can be spatial,
temporal, or spatio-temporal. By sharing the random effects between the two process, the joint
model can detect the preferential sampling effects.

We consider a spatio-temporal environmental process $Z_{st}$, $s\in \mcS$, $t\in \mcT$.
The space-time point is defined $(s, t) \in \mcS \times \mcT$, where $\mcS$ denoting the spatial 
domain of interest and $\mcT$ the temporal domain. 
Spatial network designer specifies a set of time points $T \subset \mcT$ at which to observe
$Z$ and at each time $t \in T$, a finite subset of sites $S_t \subset \mcS$ at which to do so.

$R_{st} \in \{0, 1\}$ is a binary response for the site selection process.
A Bayesian model is introduced for the joint distribution of the response vector $(Y_{st}, R_{st})$.
By sharing random effects across the two processes, the stochastic dependence (if any) between 
$Y_{s, t}$ and $R_{s, t}$ and be quantified.


\section{The joint model}
We let $Y_i(t)$ denote the spatio-temporal observation process at site $i$, that is at locations 
$s_i \in \mcP \subset \mcS$, at time $t \in T$. We let $R_i(t)$ denote the random selection 
indicator for site $s_i \in \mcP$ at time $t$. We let $t_1, \ldots, t_N$ denote the $N$ observation
times, and let $r_{i, j} \in \{0, 1\}$ denote the realization of $R_i(t_j)$, for $i \in \{1, \ldots, M\}$,
$j \in \{1, \ldots, N\}$, where $M = |\mcP|$. The general model framework is
\[
Y_{i,j} \given R_{i, j} = 1 &\sim f_{Y}(\mu_{i,j}, \theta_Y), \quad f_Y \sim \text{density},\\
g(\mu_{i, j}) = \eta_{i, j} &= x_{i, j}^T \gamma + \sum_{k=1}^{q_1}u_{i, j, k}\beta_k(s_i, t_j), 
\label{eq:gen_y} \\
R_{i, j} &\sim \distBern(p_{i, j}), \\
h(p_{i, j}) &= \nu_{i, j} = v_{i, j}^T \alpha + \sum_{\ell =1}^{q_2}d_{\ell}\sum_{k=1}^{q_1}
w_{i, j, \ell, k}\beta_k(s_i, \phi_{i, \ell, k}(t_j)) \label{eq:gen_r}\\
& + \sum_{m=1}^{q_3} w_{i, j, m}^{\star}\beta_m^{\star}(s_i, t_j), \\
\beta_k(s_i, t_j) &\sim \text{(possibly shared) latent effect with parameters}\ \theta_k, \\
& k\in \{1, \ldots, q_1\}, \\
\beta_m^{\star}(s_i, t_j) &\sim \text{site selection only latent effect with parameters}\ \theta_m^\star, \\
& m \in \{1, \ldots, q_3\}, \\
\Theta &= (\theta_Y, \alpha, \gamma, d, \theta_1, \ldots, \theta_{q_1}, \theta_1^{\star}, \ldots, \theta_{q_3}^{\star}) \sim \text{Priors}, \\
x_{i, j} \in \reals^{p_1}, u_{i, j} &\in \reals^{q_1}, v_{i, j} \in \reals^{p_2}, 
W_{i, j} \in \reals^{q_2\times q_1}, w_{i, j}^{\star T} \in \reals^{q_3} 
\]

In the linear predictor
$\eta_{i, j}$, we include a linear combination of fixed covariates $x_{i, j}$ with 
a linear combination of $q_1$ latent effects $\beta_k(s_i, t_j)$. These $q_1$ random effects
can include any combinations of spatially-correlated processes, temporally correlated processes, 
spatial temporal processes and IID random effects. Note that we include the additional fixed
covariates $u_{i, j}$ to allow for
spatially-varying coefficient models, as well as both random slopes and/or scaled random effects.

As for the site selection process $R_{i, j}$, the linear predictor $\nu_{i, j}$ may also include 
a linear combination of fixed covariates $v_{i, j}$ with a linear combination of latent effects.
In particular, the latent effects appearing in the observation process $Y_{i, j}$ are allowed to exist
in the linear predictor of the selection process $R_{i, j}$. The matrix $W_{i, j}$ is fixed
beforehand, and allow for $q_2$ linear combinations of the latent effects from the $Y_{i, j}$ 
process to be copied across. The parameter vector $d$ determines the degree to which each shared
latent effect affects the $R$ process and therefore measure the magnitude and direction of stochastic
dependence between the two models term-by-term. These $q_3$ latent effects are independent 
from the $Y_{i, j}$ process to exist in the linear predictor.

\subsection{A specific model for black smoke data in British}
\cite{Watson2019_pref_samp} introduced a specific model in the family to model the black smoke
data in British. Let $t_j^{\star}$ denote the $j$th time-scaled observations that lie in the interval
$[0, 1]$.

The model for the observation process is
\[
Y_{i, j} \given R_{i, j} &\sim \distNorm(\mu_{i, j}, \sigma_{\epsilon}^2) \\
\mu_{i, j} &= \gamma_0 + \gamma_1 t_j^{\star} + \gamma_2(t_j^{\star})^2 
+ b_{0, i} + b_{1, i}t_j^{\star} 
+ \beta_0(s_i) + \beta_1(s_i)t_j^{\star} + \beta_2(s_i)(t_j^{\star})^2 \label{eq:bs_y}\\
[\beta_k(s_1), \ldots, \beta_k(s_m)]^T &\distiid \distNorm(0, \Sigma(\zeta_k)) \quad 
\text{for}\ k \in \{0, 1, 2\}, \quad \Sigma(\zeta_k) = \text{Matern}(\zeta_k) \\
[b_{0, i}, b_{1, i}]  & \distiid \distNorm(0, \Sigma_b), \quad 
\Sigma_b = \bmat \sigma_{b, 1}^2 & \rho_b  \\ \rho_b & \sigma_{b, 2}^2 \emat, \\ 
\theta &= (\sigma_{\epsilon}^2, \gamma, \zeta_k, \sigma_{b, 1}^2, \rho_b) \sim \text{Priors}.
\]

The model for site-selection process is 
\[
R_{i, j} &\sim \distBern(p_{i, j}) \\
\mathrm{logit}\, p_{i, 1} &= \alpha_{0, 0} + \alpha_1 t_1^{\star} + \alpha_2(t_1^{\star}) 
+ \beta_1^{\star}(t_1)  \\
& + \alpha_{rep} I_{i, 2} + \beta_0^{\star}(s_i)  \\
& + d_b[b_{0, i} + b_{1, i}(t_1^{\star})] \\
& + d_{\beta}[\beta_0(s_i) + \beta_1(s_i)t_{j-1}^{\star} + \beta_2(s_i)(t_{j-1}^{\star})^2], \\
\mathrm{for} j \ne 1 \quad \mathrm{logit}\, p_{i, j} &= \alpha_{0, 1} + \alpha_1 t_j^{\star} + 
\alpha_2 (t_j^{\star})^2 + \beta_1^{\star} t_j \\
& + \alpha_{ret}r_{i, (j-1)} + \alpha_{rep} I_{i, 2} + \beta_0^{\star}(s_i)  \\
& + d_b[b_{0, i} + b_{1, i}(t_1^{\star})] \\
& + d_{\beta}[\beta_0(s_i) + \beta_1(s_i)t_{j-1}^{\star} + \beta_2(s_i)(t_{j-1}^{\star})^2], \label{eq:bs_r}\\
I_{i, j} &= \ind\left[ \left( \sum_{\ell\ne i}r_{\ell, j-1}\ind (\|s_i - s_{\ell}\| < c) \right) > 0 \right],\\
[\beta_0^{\star}(s_1), \ldots, \beta_0^{\star}(s_m)]^T &\sim \distNorm(0, \Sigma(\zeta_R)),
\Sigma(\zeta_R) = \mathrm{Matern}(\zeta_R), \\
[\beta_1^{\star}(t_1), \ldots, \beta_1^{\star}(t_T)]^T &\sim \mathrm{AR1}(\rho_a, \sigma_a^2), \\
\theta_R = [\alpha, d_b, d_{\beta}, \rho_a, \sigma_a^2, \zeta_R] & \sim \mathrm{Priors}
\]

\section{The implementation in INLA / inlabru}
To implement the preferential sampling model defined by \cref{eq:gen_y} and \cref{eq:gen_r} 
in INLA, or \textbf{inlabru}, we  are supposed to specify two models. One for the observation 
model in the Gaussian family and one for the site selection model in the Bernoulli family.
In particular, we want to share some latent factors between the observation model and the site 
selection model:
\[
\sum_{\ell =1}^{q_2}d_{\ell}\sum_{k=1}^{q_1}
w_{i, j, \ell, k}\beta_k(s_i, \phi_{i, \ell, k}(t_j)),
\]
where we have $q_2$ set of factors to share, and each set is a linear combination of $q_1$ factors
from the observation model. While both INLA and \textbf{inlabru} allow copying factors, each
factor (or component in \textbf{inlabru}) must be copied separately and therefore introduce one 
scale parameter for each copied factor (by setting $fixed = FALSE$). 

In our model, however, we only want one scale parameter for the every set of factors, i.e., we only 
want $q_2$ scale parameters in copying  factors from the observation model to the site selection 
model. For example in the specific model \cref{eq:bs_y} and \cref{eq:bs_r} for the black smoke data, the two set of copied factors in \cref{eq:bs_r}are 
\[
d_b[b_{0, i} + b_{1, i}(t_1^{\star})]  \quad \text{and }\quad d_{\beta}[\beta_0(s_i) + \beta_1(s_i)t_{j-1}^{\star} + \beta_2(s_i)(t_{j-1}^{\star})^2],
\]
where $d_b$ and $d_{\beta}$ are two scale parameters for the two set of copied factors.
If we use the $copy$ feature in INLA or \textbf{inlabru} to copy each feature separately, there will
be five (instead of two) scale parameters for each site and time point. 

To copy the whole set of factors in implementing the model for black smoke data, Watson's code
introduced two auxiliary variables and two auxiliary Gaussian models in addition to the 
observation model and the site selection model. Specifically the two auxiliary Gaussian models are
\[
0 &= - C_b +  [b_{0, i} + b_{1, i}(t_1^{\star})] \\
0 &= - C_\beta + [\beta_0(s_i) + \beta_1(s_i)t_{j-1}^{\star} + \beta_2(s_i)(t_{j-1}^{\star})^2]
\]
where $C_b$ and $C_\beta$ are auxiliary factors. To make sure that inside INLA,
\[
C_b =  b_{0, i} + b_{1, i}(t_1^{\star})\quad
\text{and} \quad
C_\beta = \beta_0(s_i) + \beta_1(s_i)t_{j-1}^{\star} + \beta_2(s_i)(t_{j-1}^{\star})^2
\]
the precision parameter of the two factors $C_b$ and $C_\beta$ are set to be $\approx 0$ and 
the precision parameter of the two Gaussian auxiliary models are set to be $\approx \infty$. 
Also, these factors $ b_{0, i}, b_{1, i}(t_1^{\star}), \beta_0(s_i), \beta_1(s_i)t_{j-1}^{\star}, \beta_2(s_i)(t_{j-1}^{\star})^2$ are copied from the observation model \cref{eq:bs_y} to the 
two auxiliary models with $fixed = TRUE$. 
The choice of the precision parameters make sure that $C_b$ and $C_\beta$ can copy these 
two set of factors precisely.

Back to the site selection model given be \cref{eq:bs_r}, instead of copying these two set of
parameters, we can copy $C_b$ and $C_{\beta}$ respectively. Also, scale parameter $d_b$
and $d_{\beta}$ are added to $C_b$ and $C_{\beta}$ respectively.

\section{The package for preferential sampling}
In , we develop an R package for this joint model framework for the purpose of making
spatial predictions. The purpose is to use the input syntax of \textbf{lme4}.

\clearpage
	
\bibliographystyle{apalike}
\bibliography{spatial_time}
\end{document}