\documentclass{article}
\usepackage[margin=1in]{geometry}
\usepackage[autonum]{tchdr}
\usepackage[sort&compress]{natbib}


\title{Copy Multiple Features in INLA / inlabru}
%\author{Xinglong Li \\xinglong.li@stat.ubc.ca}

\begin{document}
\maketitle

\section{The package for preferential sampling}
We are currently working to develop an R package for the preferential sampling model proposed
by \cite{Watson2019_pref_samp} which fits an observational model and a site selection model
that shares latent factors with the observation model. 
The purpose of this package is to facilitate spatial prediction using the proposed preferential
sampling model.

Since the joint model is restricted to two mixed effects models (one for the observation process 
and one for the site-selection process),  we would like to restrict the input of the user to simplify the API. 
In particular, we want the user to specify only formulas of the two models in addition to the dataset. 
Given that the two models are both mixed effects models, we would like to use the syntax
analogous to the that of the \textbf{lme4} package.

Internally, we want to convert the input of the user to proper models of \textbf{inlabru} and fit
the model using \textbf{inlabru}.

%\section{Background}
%\cite{Watson2019_pref_samp} proposed a framework that jointly modeling the distribution of an
%environmental process and a site-selection process, where the environmental process can be spatial,
%temporal, or spatio-temporal. By sharing the random effects between the two process, the joint
%model can detect the preferential sampling effects.
%
%We consider a spatio-temporal environmental process $Y_{st}$, $s\in \mcS$, $t\in \mcT$.
%The space-time point is defined $(s, t) \in \mcS \times \mcT$, where $\mcS$ denoting the spatial 
%domain of interest and $\mcT$ the temporal domain. 
%Spatial network designer specifies a set of time points $T \subset \mcT$ at which to observe
%$Y$ and at each time $t \in T$, a finite subset of sites $S_t \subset \mcS$ at which to do so.
%
%$R_{st} \in \{0, 1\}$ is a binary response for the site selection process.
%A Bayesian model is introduced for the joint distribution of the response vector $(Y_{st}, R_{st})$.
%By sharing random effects across the two processes, the stochastic dependence (if any) between 
%$Y_{s, t}$ and $R_{s, t}$ and be quantified.


%\section{The joint model}
%We let $Y_i(t)$ denote the spatio-temporal observation process at site $i$, that is at locations 
%$s_i \in \mcP \subset \mcS$, at time $t \in T$. We let $R_i(t)$ denote the random selection 
%indicator for site $s_i \in \mcP$ at time $t$. We let $t_1, \ldots, t_N$ denote the $N$ observation
%times, and let $r_{i, j} \in \{0, 1\}$ denote the realization of $R_i(t_j)$, for $i \in \{1, \ldots, M\}$,
%$j \in \{1, \ldots, N\}$, where $M = |\mcP|$. The general model framework is
%\[
%Y_{i,j} \given R_{i, j} = 1 &\sim f_{Y}(\mu_{i,j}, \theta_Y), \quad f_Y \sim \text{density},\\
%g(\mu_{i, j}) = \eta_{i, j} &= x_{i, j}^T \gamma + \sum_{k=1}^{q_1}u_{i, j, k}\beta_k(s_i, t_j), 
%\label{eq:gen_y} \\
%R_{i, j} &\sim \distBern(p_{i, j}), \\
%h(p_{i, j}) &= \nu_{i, j} = v_{i, j}^T \alpha + \sum_{\ell =1}^{q_2}d_{\ell}\sum_{k=1}^{q_1}
%w_{i, j, \ell, k}\beta_k(s_i, \phi_{i, \ell, k}(t_j)) \label{eq:gen_r}\\
%& + \sum_{m=1}^{q_3} w_{i, j, m}^{\star}\beta_m^{\star}(s_i, t_j), \\
%\beta_k(s_i, t_j) &\sim \text{(possibly shared) latent effect with parameters}\ \theta_k, \\
%& k\in \{1, \ldots, q_1\}, \\
%\beta_m^{\star}(s_i, t_j) &\sim \text{site selection only latent effect with parameters}\ \theta_m^\star, \\
%& m \in \{1, \ldots, q_3\}, \\
%\Theta &= (\theta_Y, \alpha, \gamma, d, \theta_1, \ldots, \theta_{q_1}, \theta_1^{\star}, \ldots, \theta_{q_3}^{\star}) \sim \text{Priors}, \\
%x_{i, j} \in \reals^{p_1}, u_{i, j} &\in \reals^{q_1}, v_{i, j} \in \reals^{p_2}, 
%W_{i, j} \in \reals^{q_2\times q_1}, w_{i, j}^{\star T} \in \reals^{q_3} 
%\]
%
%In the linear predictor
%$\eta_{i, j}$, we include a linear combination of fixed covariates $x_{i, j}$ with 
%a linear combination of $q_1$ latent effects $\beta_k(s_i, t_j)$. These $q_1$ random effects
%can include any combinations of spatially-correlated processes, temporally correlated processes, 
%spatial temporal processes and IID random effects. Note that we include the additional fixed
%covariates $u_{i, j}$ to allow for
%spatially-varying coefficient models, as well as both random slopes and/or scaled random effects.
%
%As for the site selection process $R_{i, j}$, the linear predictor $\nu_{i, j}$ may also include 
%a linear combination of fixed covariates $v_{i, j}$ with a linear combination of latent effects.
%In particular, the latent effects appearing in the observation process $Y_{i, j}$ are allowed to exist
%in the linear predictor of the selection process $R_{i, j}$. The matrix $W_{i, j}$ is fixed
%beforehand, and allow for $q_2$ linear combinations of the latent effects from the $Y_{i, j}$ 
%process to be copied across. The parameter vector $d$ determines the degree to which each shared
%latent effect affects the $R$ process and therefore measure the magnitude and direction of stochastic
%dependence between the two models term-by-term. These $q_3$ latent effects are independent 
%from the $Y_{i, j}$ process to exist in the linear predictor.

\section{A preferential sampling model for black smoke data in British}
We consider a spatio-temporal environmental process $Y_{st}$, $s\in \mcS$, $t\in \mcT$, 
where $\mcS$ denoting the spatial domain of interest and $\mcT$ the temporal domain. 
Spatial network designer specifies a set of time points $T \subset \mcT$ at which to observe
$Y$ and at each time $t \in T$, a finite subset of sites $S_t \subset \mcS$ at which to do so.
$R_{st} \in \{0, 1\}$ is a binary response for the site selection process.
A Bayesian model is introduced for the joint distribution of the response vector $(Y_{st}, R_{st})$.

By sharing random effects across the two processes, the stochastic dependence (if any) between 
$Y_{s, t}$ and $R_{s, t}$ and be quantified.
\cite{Watson2019_pref_samp} proposed one such preferential sampling model to analyze the 
black smoke data in British. Let $t_j^{\star}$ denote the $j$th time-scaled observations that lie 
in the interval $[0, 1]$.

The model for the observation process is
\[
Y_{i, j} \given R_{i, j} &\sim \distNorm(\mu_{i, j}, \sigma_{\epsilon}^2) \\
\mu_{i, j} &= \gamma_0 + \gamma_1 t_j^{\star} + \gamma_2(t_j^{\star})^2 
+ b_{0, i} + b_{1, i}t_j^{\star} 
+ \beta_0(s_i) + \beta_1(s_i)t_j^{\star} + \beta_2(s_i)(t_j^{\star})^2 \label{eq:bs_y}\\
[\beta_k(s_1), \ldots, \beta_k(s_m)]^T &\distiid \distNorm(0, \Sigma(\zeta_k)) \quad 
\text{for}\ k \in \{0, 1, 2\}, \quad \Sigma(\zeta_k) = \text{Matern}(\zeta_k) \\
[b_{0, i}, b_{1, i}]  & \distiid \distNorm(0, \Sigma_b), \quad 
\Sigma_b = \bmat \sigma_{b, 1}^2 & \rho_b  \\ \rho_b & \sigma_{b, 2}^2 \emat, \\ 
\theta &= (\sigma_{\epsilon}^2, \gamma, \zeta_k, \sigma_{b, 1}^2, \rho_b) \sim \text{Priors}.
\]

The model for site-selection process is 
\[
R_{i, j} &\sim \distBern(p_{i, j}) \\
\mathrm{logit}\, p_{i, 1} &= \alpha_{0, 0} + \alpha_1 t_1^{\star} + \alpha_2(t_1^{\star}) 
+ \beta_1^{\star}(t_1)  \\
& + \alpha_{rep} I_{i, 2} + \beta_0^{\star}(s_i)  \\
& + d_b[b_{0, i} + b_{1, i}(t_1^{\star})] \\
& + d_{\beta}[\beta_0(s_i) + \beta_1(s_i)t_{j-1}^{\star} + \beta_2(s_i)(t_{j-1}^{\star})^2], \\
\mathrm{for} j \ne 1 \quad \mathrm{logit}\, p_{i, j} &= \alpha_{0, 1} + \alpha_1 t_j^{\star} + 
\alpha_2 (t_j^{\star})^2 + \beta_1^{\star} t_j \\
& + \alpha_{ret}r_{i, (j-1)} + \alpha_{rep} I_{i, 2} + \beta_0^{\star}(s_i)  \\
& + d_b[b_{0, i} + b_{1, i}(t_1^{\star})] \\
& + d_{\beta}[\beta_0(s_i) + \beta_1(s_i)t_{j-1}^{\star} + \beta_2(s_i)(t_{j-1}^{\star})^2], \label{eq:bs_r}\\
I_{i, j} &= \ind\left[ \left( \sum_{\ell\ne i}r_{\ell, j-1}\ind (\|s_i - s_{\ell}\| < c) \right) > 0 \right],\\
[\beta_0^{\star}(s_1), \ldots, \beta_0^{\star}(s_m)]^T &\sim \distNorm(0, \Sigma(\zeta_R)),
\Sigma(\zeta_R) = \mathrm{Matern}(\zeta_R), \\
[\beta_1^{\star}(t_1), \ldots, \beta_1^{\star}(t_T)]^T &\sim \mathrm{AR1}(\rho_a, \sigma_a^2), \\
\theta_R = [\alpha, d_b, d_{\beta}, \rho_a, \sigma_a^2, \zeta_R] & \sim \mathrm{Priors}
\]

The latent effects appearing in the observation process $Y_{i, j}$ are allowed to exist
in the linear predictor of the selection process $R_{i, j}$. In particular, the two linear combinations 
of the latent effects, $b_{0, i} + b_{1, i}(t_1^{\star})$ and 
$\beta_0(s_i) + \beta_1(s_i)t_{j-1}^{\star} + \beta_2(s_i)(t_{j-1}^{\star})^2$, from the $Y_{i, j}$ 
process are copied across. The parameters $d_b$ and $d_{\beta}$ determine the degree to which each shared
latent effect affects the $R$ process and therefore measure the magnitude and direction of stochastic
dependence between the two models term-by-term. 

\section{The implementation in INLA / inlabru}
To implement the preferential sampling model defined by \cref{eq:bs_y} and \cref{eq:bs_r} 
in INLA, or \textbf{inlabru}, we  are supposed to specify two models. One for the observation 
process in the Gaussian family and one for the site selection process in the Bernoulli family.
Also, we want to share two linear combinations of latent factors between the observation
model and the site selection model:
\[
b_{0, i} + b_{1, i}(t_1^{\star}), \quad \text{and}\quad \beta_0(s_i) + \beta_1(s_i)t_{j-1}^{\star} + \beta_2(s_i)(t_{j-1}^{\star})^2.
\]

While both INLA and \textbf{inlabru} allow copying factors 
between models, each factor (`component' in \textbf{inlabru}) must be copied separately and
therefore introduce one new scale parameter for each copied factor (by setting $fixed = FALSE$). 
In our model, however, 
we only  want two scale parameters $d_b$ and $d_\beta$ for these two linear combinations
of factors:
\[
d_b[b_{0, i} + b_{1, i}(t_1^{\star})]  \quad \text{and }\quad d_{\beta}[\beta_0(s_i) + \beta_1(s_i)t_{j-1}^{\star} + \beta_2(s_i)(t_{j-1}^{\star})^2],
\]
where $d_b$ and $d_{\beta}$ are two scale parameters.
This is not directly achievable using the $copy$ feature in INLA or \textbf{inlabru},
and if we use the $copy$ feature to copy each latent factor separately, 
there will be five (instead of two) new scale parameters introduced at each site and time point. 

\subsection{An alternative approach using auxiliary models}
To copy the linear combinations of factors in implementing the model for black smoke data, \cite{Watson2019_pref_samp}
introduced two auxiliary factors and two auxiliary Gaussian models in addition to the 
original joint model:
\[
0 &= - C_b +  [b_{0, i} + b_{1, i}(t_1^{\star})] \label{eq:bs_aux0} \\
0 &= - C_\beta + [\beta_0(s_i) + \beta_1(s_i)t_{j-1}^{\star} + \beta_2(s_i)(t_{j-1}^{\star})^2] 
\label{eq:bs_aux1}
\]
where $C_b$ and $C_\beta$ are auxiliary latent factors. These individual factors, $b_{0, i}$, $b_{1, i}(t_1^{\star})$, $\beta_0(s_i)$, 
$\beta_1(s_i)t_{j-1}^{\star}$, $\beta_2(s_i)(t_{j-1}^{\star})^2$, are copied separately from the 
observation model \cref{eq:bs_y} to the  two auxiliary models, \cref{eq:bs_aux0} and 
\cref{eq:bs_aux1}, with the argument $fixed = TRUE$. 

By setting the precision parameter of the two factors $C_b$ and $C_\beta$ to be $\approx 0$ and 
setting the precision parameter of the two Gaussian auxiliary models to be $\approx \infty$,
the latent factors $C_b$ and $C_\beta$ duplicate of the two factor combinations:
\[
C_b =  b_{0, i} + b_{1, i}(t_1^{\star})\quad
\text{and} \quad
C_\beta = \beta_0(s_i) + \beta_1(s_i)t_{j-1}^{\star} + \beta_2(s_i)(t_{j-1}^{\star})^2.
\] 
Given the two auxiliary models, the new model for site-selection process copies $C_b$ 
and $C_\beta$ from \cref{eq:bs_aux0} and \cref{eq:bs_aux1} instead with the argument $fixed = FALSE$:
\[
\mathrm{logit}\, p_{i, 1} &= \alpha_{0, 0} + \alpha_1 t_1^{\star} + \alpha_2(t_1^{\star}) 
+ \beta_1^{\star}(t_1)  \\
& + \alpha_{rep} I_{i, 2} + \beta_0^{\star}(s_i)  \\
& + d_b C_b + d_{\beta}C_\beta, \\
\mathrm{for} j \ne 1 \quad \mathrm{logit}\, p_{i, j} &= \alpha_{0, 1} + \alpha_1 t_j^{\star} + 
\alpha_2 (t_j^{\star})^2 + \beta_1^{\star} t_j \\
& + \alpha_{ret}r_{i, (j-1)} + \alpha_{rep} I_{i, 2} + \beta_0^{\star}(s_i)  \\
& + d_b C_b + d_{\beta}C_{\beta}.
\]

With the auxiliary models and factors, it is possible to copy the linear combination of factors 
without introducing too many scale parameters. 
However, this approach requires us to fit four, instead of two models in INLA(or \textbf{inlabru}), 
and in general, more auxiliary models and factors will be required if more linear combinations 
of factors need to be shared between the observation process and the site selection process.

\section{Question}
To simplify the API of our package so that users can use the preferential sampling model to 
make spatial predictions easily, we want to follow the syntax of the package \textbf{lme4} and 
ask users to only provide formulas of two mixed effects models. Inside the package, we need to 
convert the joint model to \textbf{inlabru}(or INLA) models.  Since the approach used by 
\cite{Watson2019_pref_samp} requires one more additional model for each linear combination 
of factors to be copied, this increases the complexity of the code. So we wonder if there is more
straightforward way to copy multiple / linear combination of factors  across models in INLA or 
\textbf{inlabru}.
	
\bibliographystyle{apalike}
\bibliography{spatial_time}
\end{document}