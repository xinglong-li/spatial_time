\section{The implementation using inlabru}
To implement the preferential sampling model defined by \cref{eq:bs_y} and \cref{eq:bs_r} 
in INLA, or \textbf{inlabru}, we  are supposed to specify two models. One for the observation 
process in the Gaussian family and one for the site selection process in the Bernoulli family.
Also, we want to share two linear combinations of latent factors between the observation
model and the site selection model:
\[
b_{0, i} + b_{1, i}(t_1^{\star}), \quad \text{and}\quad \beta_0(s_i) + \beta_1(s_i)t_{j-1}^{\star} + \beta_2(s_i)(t_{j-1}^{\star})^2.
\]

While both INLA and \textbf{inlabru} allow copying factors 
between models, each factor (`component' in \textbf{inlabru}) must be copied separately and
therefore introduce one new scale parameter for each copied factor (by setting $fixed = FALSE$). 
In our model, however, 
we only  want two scale parameters $d_b$ and $d_\beta$ for these two linear combinations
of factors:
\[
d_b[b_{0, i} + b_{1, i}(t_1^{\star})]  \quad \text{and }\quad d_{\beta}[\beta_0(s_i) + \beta_1(s_i)t_{j-1}^{\star} + \beta_2(s_i)(t_{j-1}^{\star})^2],
\]
where $d_b$ and $d_{\beta}$ are two scale parameters.
This is not directly achievable using the $copy$ feature in INLA or \textbf{inlabru},
and if we use the $copy$ feature to copy each latent factor separately, 
there will be five (instead of two) new scale parameters introduced at each site and time point. 

\subsection{An alternative approach using auxiliary models}
To copy the linear combinations of factors in implementing the model for black smoke data, \cite{Watson2019_pref_samp}
introduced two auxiliary factors and two auxiliary Gaussian models in addition to the 
original joint model:
\[
0 &= - C_b +  [b_{0, i} + b_{1, i}(t_1^{\star})] \label{eq:bs_aux0} \\
0 &= - C_\beta + [\beta_0(s_i) + \beta_1(s_i)t_{j-1}^{\star} + \beta_2(s_i)(t_{j-1}^{\star})^2] 
\label{eq:bs_aux1}
\]
where $C_b$ and $C_\beta$ are auxiliary latent factors. These individual factors, $b_{0, i}$, $b_{1, i}(t_1^{\star})$, $\beta_0(s_i)$, 
$\beta_1(s_i)t_{j-1}^{\star}$, $\beta_2(s_i)(t_{j-1}^{\star})^2$, are copied separately from the 
observation model \cref{eq:bs_y} to the  two auxiliary models, \cref{eq:bs_aux0} and 
\cref{eq:bs_aux1}, with the argument $fixed = TRUE$. 

By setting the precision parameter of the two factors $C_b$ and $C_\beta$ to be $\approx 0$ and 
setting the precision parameter of the two Gaussian auxiliary models to be $\approx \infty$,
the latent factors $C_b$ and $C_\beta$ duplicate of the two factor combinations:
\[
C_b =  b_{0, i} + b_{1, i}(t_1^{\star})\quad
\text{and} \quad
C_\beta = \beta_0(s_i) + \beta_1(s_i)t_{j-1}^{\star} + \beta_2(s_i)(t_{j-1}^{\star})^2.
\] 
Given the two auxiliary models, the new model for site-selection process copies $C_b$ 
and $C_\beta$ from \cref{eq:bs_aux0} and \cref{eq:bs_aux1} instead with the argument $fixed = FALSE$:
\[
\mathrm{logit}\, p_{i, 1} &= \alpha_{0, 0} + \alpha_1 t_1^{\star} + \alpha_2(t_1^{\star}) 
+ \beta_1^{\star}(t_1)  \\
& + \alpha_{rep} I_{i, 2} + \beta_0^{\star}(s_i)  \\
& + d_b C_b + d_{\beta}C_\beta, \\
\mathrm{for} j \ne 1 \quad \mathrm{logit}\, p_{i, j} &= \alpha_{0, 1} + \alpha_1 t_j^{\star} + 
\alpha_2 (t_j^{\star})^2 + \beta_1^{\star} t_j \\
& + \alpha_{ret}r_{i, (j-1)} + \alpha_{rep} I_{i, 2} + \beta_0^{\star}(s_i)  \\
& + d_b C_b + d_{\beta}C_{\beta}.
\]

With the auxiliary models and factors, it is possible to copy the linear combination of factors 
without introducing too many scale parameters. 
However, this approach requires us to fit four, instead of two models in INLA(or \textbf{inlabru}), 
and in general, more auxiliary models and factors will be required if more linear combinations 
of factors need to be shared between the observation process and the site selection process.
