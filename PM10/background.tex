\section{Background}
We consider a spatio-temporal environmental process $Z_{st}$, $s\in \mcS$, $t\in \mcT$.
The space-time point is defined $(s, t) \in \mcS \times \mcT$, where $\mcS$ denoting the spatial 
domain of interest and $\mcT$ the temporal domain. 
Spatial network designer must specify a set of time points $T \subset \mcT$ at which to observe
$Z$ and at each time $t \in T$, a finite subset of sites $S_t \subset \mcS$ at which to do so.

The population of all site locations considered for selection at any time $t \in T$ is defined as 
$\mcP \subset \mcS$, and $\mcP$ is finite and should be specified a priori. 
A Bayesian model is introduced for the joint distribution of the response vector $(Y_{st}, R_{st})$.
$R_{st} \in \{0, 1\}$ is a binary response for the site selection process.
By sharing random effects across the two processes, the stochastic dependence (if any) between 
$Y_{s, t}$ and $R_{s, t}$ and be quantified and subsequently the model can adjust the space-time
predictions according to the preferential sampling effect detected.
Furthermore, in the joint model, the factors affecting the initial site placement can be allowed to
differ from those affecting the retention of existing sites in the network.

\subsection{The joint model}
We let $Y_i(t)$ denote the spatio-temporal observation process at site $i$, that is at locations 
$s_i \in \mcP \subset \mcS$, at time $t \in T$. We let $R_i(t)$ denote the random selection 
indicator for site $s_i \in \mcP$ at time $t$. We let $t_1, \ldots, t_N$ denote the $N$ observation
times, and let $r_{i, j} \in \{0, 1\}$ denote the realization of $R_i(t_j)$, for $i \in \{1, \ldots, M\}$,
$j \in \{1, \ldots, N\}$, where $M = |\mcP|$. The general model framework is
\[
(Y_{i,j} \given R_{i, j} = 1) &\sim f_{Y}(\mu_{i,j}, \theta_Y), \quad f_Y \sim \text{density},\\
g(\mu_{i, j}) = \eta_{i, j} &= \mathbf{x}_{i, j}^T \gamma + \sum_{k=1}^{q_1}u_{i, j, k}\beta_k(s_i, t_j), \\
R_{i, j} &\sim \distBern(p_{i, j}), \\
h(p_{i, j}) &= \nu_{i, j} = \mathbf{v}_{i, j}^T \alpha + \sum_{\ell =1}^{q_2}d_{\ell}\sum_{k=1}^{q_1}
w_{i, j, \ell, k}\beta_k(s_i, \phi_{i, \ell, k}(t_j)) \\
& + \sum_{m=1}^{q_3} w_{i, j, m}^{\star}\beta_m^{\star}(s_i, t_j), \\
\beta_k(s_i, t_j) &\sim \text{(possibly shared) latent effect with parameters}\ \theta_k, \\
& k\in \{1, \ldots, q_1\}, \\
\beta_m^{\star}(s_i, t_j) &\sim \text{site selection only latent effect with parameters}\ \theta_m^\star, \\
& m \in \{1, \ldots, q_3\}, \\
\Theta &= (\theta_Y, \alpha, \gamma, d, \theta_1, \ldots, \theta_{q_1}, \theta_1^{\star}, \ldots, \theta_{q_3}^{\star}) \sim \text{Priors}, \\
\mathbf{x}_{i, j} \in \reals^{p_1}, \mathbf{u}_{i, j} &\in \reals^{q_1}, \mathbf{v}_{i, j} \in \reals^{p_2}, 
\mathbf{W}_{i, j} \in \reals^{q_2\times q_1}, \mathbf{w}_{i, j}^{\star T} \in \reals^{q_3} 
\]

This framework allows a range of different data types of $Y$ to be modeled. In the linear predictor
$\eta_{i, j}$, we include a linear combination of fixed covariates $\bfx_{i, j}$ with 
a linear combination of $q_1$ latent effects $\beta_k(s_i, t_j)$. These $q_1$ random effects
can include any combinations of spatially-correlated processes (such as Gaussian [Markov] random
fields), temporally correlated processes (such as autoregressive terms), spatial temporal processes
and IID random effects. Note that we include the additional fixed covariates $\bfu_{i, j}$ to allow for
spatially-varying coefficient models, as well as both random slopes and/or scaled random effects.

As for the site selection process $R_{i, j}$, the linear predictor $\nu_{i, j}$ may also include 
a linear combination of fixed covariates $\bfv_{i, j}$ with a linear combination of latent effects.
In particular, the latent effects appearing in the observation process $Y_{i, j}$ are allowed to exist
in the linear predictor of the selection process $R_{i, j}$. Note that the matrix $\bfW_{i, j}$ is fixed
beforehand, and allow for $q_2$ linear combinations of the latent effects from the $Y_{i, j}$ 
process to be copied across. The parameter vector $\bfd$ determines the degree to which each shared
latent effect affects the $\bfR$ process and therefore measure the magnitude and direction of stochastic
dependence between the two models term-by-term. We allow $q_3$ latent effects, independent 
from the $Y_{i, j}$ process to exist in the linear predictor.

For added flexibility we allow temporal lags in the stochastic dependence. This allows the 
site-selection process to depend on the realized values of the latent effects at any time arbitrary 
time in the past, present or future. For example, if for a pollution monitoring network, 
site-selection were desired near immediate sources of pollution, then we may view as reasonable, 
a model that allows for a dependence between the latent field at the previous time step 
as a site-selection emulator. In this case, we would select as temporal lag function 
$\phi_{i, \ell, k}(t_j) = t_{j-1}$. 

Also of interest is the possibility of setting $w_{i,j,\ell,m} = 0$ for some values of the subscripts
to allow for the directions of preferentiality to change through time. For example, 
the initial placement of the sites might be made in a positively (or negatively) preferential manner
but over time the network might be redesigned so that sites were later placed to reduce the bias. 
To capture this, it would make sense to have a separate PS parameter $d$ estimated for time $t = 1$
and for times $t > 1$ to capture the changing directions of preferentiality through time. 
This can easily be implemented. Furthermore, we may wish to set $w_{i,j,\ell,m} = 0$ for certain 
values of the subscripts to see if the effects of covariates and/or the effects of preferential sampling 
differs between the initial site placement process and the site retention process.

\subsection{A specific model}
We build one model from the general framework introduced earlier. Let $t_j^{\star}$ denote 
the $j$th time-scaled observations that lie in the interval $[0, 1]$.
The model for the observation process is
\[
(Y_{i, j} \given R_{i, j}=1) &\sim \distNorm(\mu_{i, j}, \sigma_{\epsilon}^2) \\
\mu_{i, j} &= \gamma_0 + \gamma_1 t_j^{\star} + \gamma_2(t_j^{\star})^2 
+ b_{0, i} + b_{1, i}t_j^{\star} 
+ \beta_0(s_i) + \beta_1(s_i)t_j^{\star} + \beta_2(s_i)(t_j^{\star})^2 \label{eq:bs_y}\\
[\beta_k(s_1), \ldots, \beta_k(s_m)]^T &\distiid \distNorm(0, \Sigma(\zeta_k)) \quad 
\text{for}\ k \in \{0, 1, 2\}, \quad \Sigma(\zeta_k) = \text{Matern}(\zeta_k) \\
[b_{0, i}, b_{1, i}]  & \distiid \distNorm(0, \Sigma_b), \quad 
\Sigma_b = \bmat \sigma_{b, 1}^2 & \rho_b  \\ \rho_b & \sigma_{b, 2}^2 \emat, \\ 
\theta &= (\sigma_{\epsilon}^2, \gamma, \zeta_k, \sigma_{b, 1}^2, \rho_b) \sim \text{Priors}.
\]
The model for site-selection process is 
\[
R_{i, j} &\sim \distBern(p_{i, j}) \\
\mathrm{logit}\, p_{i, 1} &= \alpha_{0, 0} + \alpha_1 t_1^{\star} + \alpha_2(t_1^{\star})^2 
+ \beta_1^{\star}(t_1)  \\
& + \alpha_{rep} I_{i, 2} + \beta_0^{\star}(s_i)  \\
& + d_b[b_{0, i} + b_{1, i}(t_1^{\star})] \\
& + d_{\beta}[\beta_0(s_i) + \beta_1(s_i)t_{1}^{\star} + \beta_2(s_i)(t_{1}^{\star})^2], \\
\mathrm{for} j \ne 1 \quad \mathrm{logit}\, p_{i, j} &= \alpha_{0, 1} + \alpha_1 t_j^{\star} + 
\alpha_2 (t_j^{\star})^2 + \beta_1^{\star}(t_j) \\
& + \alpha_{ret}r_{i, (j-1)} + \alpha_{rep} I_{i, j} + \beta_0^{\star}(s_i)  \\
& + d_b[b_{0, i} + b_{1, i}(t_{j-1}^{\star})] \\
& + d_{\beta}[\beta_0(s_i) + \beta_1(s_i)t_{j-1}^{\star} + \beta_2(s_i)(t_{j-1}^{\star})^2], \label{eq:bs_r}\\
I_{i, j} &= \ind\left[ \left( \sum_{\ell\ne i}r_{\ell, j-1}\ind (\|s_i - s_{\ell}\| < c) \right) > 0 \right],\\
[\beta_0^{\star}(s_1), \ldots, \beta_0^{\star}(s_m)]^T &\sim \distNorm(0, \Sigma(\zeta_R)),
\Sigma(\zeta_R) = \mathrm{Matern}(\zeta_R), \\
[\beta_1^{\star}(t_1), \ldots, \beta_1^{\star}(t_T)]^T &\sim \mathrm{AR1}(\rho_a, \sigma_a^2), \\
\theta_R = [\alpha, d_b, d_{\beta}, \rho_a, \sigma_a^2, \zeta_R] & \sim \mathrm{Priors}
\]

The first component is the global effects of time on the log odds of selection. We also add 
first-order autoregressive deviation, $\beta_1^{\star}(t_j)$, from this global quadratic change.
$\alpha_{ret}$ represents the "retention effect" reflecting how the probability a site is selected in a 
given year changes, conditioned on its inclusion in the previous year. Here, we share all parameters
across the two processes and allow only a unique intercept to exist between the processes.
$\alpha_{rep}$ captures the repulsion effect. $I_{i, j}$ denote an indicator variable that determines
whether or not another site in the network placed within a distance $c$ from site $i$ was
was operational at the previous time $t_{j-1}$. We choose the hyperparameter $c$ to be 10 km.

This is a joint model with three processes: an observation process, an initial site-placement process
and a site-retention process. We only allow for a unique intercept to exist across the two processes, 
sharing the remaining parameters. Only the pseudo-sites contribute a zero to the 
Bernoulli likelihood for the site-placement across all years. Only the sites that have been removed 
from the network in year $j$ contribute a zero to the Bernoulli likelihood for the site-retention
process at year $j$. This ensures that no site in the network was ever reinstalled after its removal. 

The latent effects appearing in the observation process $Y_{i, j}$ are allowed to exist
in the linear predictor of the selection process $R_{i, j}$. In particular, the two linear combinations 
of the latent effects, $b_{0, i} + b_{1, i}(t_1^{\star})$ and 
$\beta_0(s_i) + \beta_1(s_i)t_{j-1}^{\star} + \beta_2(s_i)(t_{j-1}^{\star})^2$, from the $Y_{i, j}$ 
process are copied across. The parameters $d_b$ and $d_{\beta}$ determine the degree to which each shared
latent effect affects the $R$ process and therefore measure the magnitude and direction of stochastic
dependence between the two models term-by-term. 
