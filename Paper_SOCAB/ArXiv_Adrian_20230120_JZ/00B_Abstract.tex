\begin{center}
    \textbf{Abstract}
\end{center}
The preferential siting of the locations of monitors of hazardous environmental fields can lead to the serious underestimation of the impacts of those fields. In particular, human health effects can be severely underestimated when standard statistical are applied without appropriate adjustment.  This report describes an extensive analysis of the siting of monitors for a network that measures air pollution $PM_{10}$ in California's South Coast Air Basin SOCAB. That analysis uses EPA data collected during the 1986 -- 2019 period. Background descriptions, including those published by the US EPA are provided. The analysis uses a very general and fast Monte Carlo test for preferential sampling developed by Dr Joe Watson, which confirms that the sites were preferentially sited, as would be expected, given the intended purpose of the network to detect noncompliance with air quality standards. Our findings demonstrate both the value of that algorithm for application where  where such background knowledge is not available, and hence to situations in which standard statistical tools require modification.

 