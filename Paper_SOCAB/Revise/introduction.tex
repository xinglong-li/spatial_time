\section{Introduction}\label{sec:introduction}

Air pollution is a continuous three-dimensional field.  It exists on many spatial scales depending upon the pollutant, from a city block to the globe.  This report focuses on ground level \ac{PM10}.  This focus simplifies the field. It becomes a two-dimensional surface, with changes being on the scale of kilometres \citep{CFR:Title40-58} instead.   

The field can only be monitored by taking point measurements and extrapolating these over the entire region of interest. The collection of monitoring sites is called a monitoring network.  That network fulfills one or more specific purposes: overall field estimation; monitoring for pollutant compliance; assessing concentrations over population centers; and forecasting.  These goals do not necessarily encompass the capture of the field's mean level, in which case the network may generate a biased assessment of the overall concentration field. This bias may not matter; if the network were meant to detect noncompliance, the sites should be located in regions most likely to be out of compliance. 

However, the data from the network may well be used for unintended applications. Since most common statistical procedures assume that sampling is not preferential, i.e. unbiased, 
applying these techniques to data can yield result in erroneous conclusions.  For example, there may 
be an inverse impact on health impact parameters: if the bias were towards high observations, the effect of pollution would be underestimated \citep{Zidek:2012}.


That leads to the study reported in the paper, which presents a way of detecting bias, if any,
in multi-level governmental networks for monitoring air quality in the United States in general and the region surrounding Los Angeles in particular.  Because the US government makes data freely available, the data are used for many purposes, some of which are unintended. An example would be epidemiological studies that attempt to link disease frequency to pollution levels.

The paper reports evidence of bias in the sampling of $PM_{10}$ in the \ac{SOCAB}.  That bias has been acknowledged as intentional by the governmental body in charge, the \ac{SCAQMD}.  Thus, it should be considered in any work that uses those networks. Furthermore, because of the bias's possible origin in policy, caution should extend to any data from these types of compliance monitoring networks.

\subsection{Motivation for the Paper}
\label{subsec:motivation}
This study set out to explore ways of detecting monitoring site selection bias, with a focus on the South California Air Basin SOCAB monitoring region after its several decades of data monitoring.  Southern California has a long history of recognizing air pollution as a problem, dating back to 1945 \citep{CASCAQMD:2015}.

The models used to describe spatial fields generally assume a random placement of monitoring sites or at least independent conditional on its latent underlying, latent field.  However, it seems the placement of monitoring sites is often not random. They are often chosen to fulfill a range of constraints. Even if the monitoring network were well-designed, sites might be chosen for termination because their local air pollution fields are consistently in compliance.  In short, selection bias, referred to as \ac{PS}, can lead to models that don't reflect the actual pollution field experienced by the population. 

Concern about \ac{PS} has a decade's long history. \citet{isaaks1988spatial} discuss how clustered data make variograms poor at estimating covariance parameters. \cite{diggle:07} define \ac{PS} in their book as the stochastic dependence of site locations upon the property being measured. \cite{shaddick2012preferential} discovered  \ac{PS} in the United Kingdom's black smoke monitoring network. Numerous other papers have examined \ac{PS} in different cases, showing how the \ac{PS} results in models incorrectly attributing the magnitude of pollution with impact on health or other model parameters.  An extensive list of references can be found in \citep{Zidek:2012}.  

The data gathered in the US is freely available to the public and so gets put to many different uses.  Government agencies use this data to make real-time air quality warnings, to monitor general compliance of regions to meet predefined standards and to monitor point sources.  Healthcare specialists use the data from the monitoring in correlational studies to predict the health impacts of pollutant levels on the general population and subsets of interest.  \cite{wong2004comparison} brought up various concerns about using different interpolation techniques with \ac{EPA} data for epidemiological studies.  This combination of circumstances led us to be curious about whether the monitoring networks of the US exhibit \ac{PS}.