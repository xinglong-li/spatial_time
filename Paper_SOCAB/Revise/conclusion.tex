%\include{07_Conclusions}
\section{Conclusions}\label{sec:conclusions}

\subsection{Presence of Preferential Sampling}
\label{subsec:presenceprefsamp}

This report found evidence of Preferential Sampling in the \ac{SOCAB}.  Government reports provided textual support, while statistical support came from both the pattern of network adjustment and the \texttt{PStestR} package.  This \ac{PS} results in observed pollution higher than the regional mean.


\subsection{Complications Not Considered}
\label{subsec:notconsidered}
Throughout this work, several ways of adding complexity were sidelined.  Also, having finished, several extensions or new angles of inquiry have occurred to me.  These include considerations when modelling the field, ways to interpret preferential sampling, and how to extend the work to applicability.

\subsubsection*{Modelling the field}
\label{subsubsec:modellingfield}
As discussed in Section \ref{subsubsec:DataRows}, many sites have multiple instruments recording \ac{PM10} at the same time.  These could provide an understanding of the nugget effect by being replicated measurements.  However, this would require modifications to the \ac{INLA} model to account for the unusual presence of information on the nugget.  Therefore, the instruments were combined into a mean to simplify modelling.

Section \ref{subsubsec:SpatialDomain} described how the study area to \ac{SOCAB} was constrained.  However, the decisions are made by the \ac{SCAQMD} which has jurisdiction over a wider area. If the study domain could be extended to the jurisdictional boundary instead of the geographic airshed boundary, a fuller understanding of the preferential sampling process would be achieved. However, this requires modelling the discontinuity in the \ac{PM10} surface.

The exploratory examination of the preferential sampling time series in Section \ref{subsubsec:TestScoreTimeSeries} did not account for the lack of independence between years.  It is not known what modifications would have to be made to the \texttt{PStestR} algorithm to account for this, but it is another area that could be examined.

\subsubsection*{Understanding Preferential Sampling}
\label{subsubsec:understandingPF}

The MCMC samples of hypothetical networks generated by \texttt{PStestR} placed sites with a uniform distribution over the SOCAB's area.  However, sites have numerous constraints on their actual real-world location.  Implementing these constraints would require finding documentation of what considerations are made for site selection, and then using a GIS tool with data layers describing those considerations.  Another way to examine the network for preferential sampling is whether a given site complies with \ac{EPA} standards.  If decisions are being made based upon a site's compliance or lack thereof, including it into the \ac{PS} model could help understand the decisions.

\subsubsection*{Future Applicability}
\label{subsubsec:futureapps}
Much of the motivation for detecting preferential sampling is from its potential impact on studies using the observed data as an unbiased sample of population exposure.  Having found evidence for Preferential Sampling, a reasonable next step could be to determine how this bias has affected various studies.  A sensitivity study testing the impact of \ac{PS} on health studies might help.
